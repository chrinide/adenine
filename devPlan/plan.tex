%----------------------------------------------------------------------------------------
%	PACKAGES AND OTHER DOCUMENT CONFIGURATIONS
%----------------------------------------------------------------------------------------

\documentclass[paper=a4, fontsize=10pt]{scrartcl} % A4 paper and 10pt font size

\usepackage[T1]{fontenc} % Use 8-bit encoding that has 256 glyphs
\usepackage[english]{babel} % English language/hyphenation
\usepackage{amsmath,amsfonts,amsthm} % Math packages

\usepackage[margin=1in]{geometry}

\usepackage{sectsty} % Allows customizing section commands
\allsectionsfont{\centering \normalfont\scshape} % Make all sections centered, the default font and small caps

\usepackage{fancyhdr} % Custom headers and footers
\pagestyle{fancyplain} % Makes all pages in the document conform to the custom headers and footers
\fancyhead{} % No page header - if you want one, create it in the same way as the footers below
\fancyfoot[L]{} % Empty left footer
\fancyfoot[C]{} % Empty center footer
\fancyfoot[R]{\thepage} % Page numbering for right footer
\renewcommand{\headrulewidth}{0pt} % Remove header underlines
\renewcommand{\footrulewidth}{0pt} % Remove footer underlines
\setlength{\headheight}{11pt} % Customize the height of the header

\numberwithin{equation}{section} % Number equations within sections (i.e. 1.1, 1.2, 2.1, 2.2 instead of 1, 2, 3, 4)
\numberwithin{figure}{section} % Number figures within sections (i.e. 1.1, 1.2, 2.1, 2.2 instead of 1, 2, 3, 4)
\numberwithin{table}{section} % Number tables within sections (i.e. 1.1, 1.2, 2.1, 2.2 instead of 1, 2, 3, 4)

\setlength\parindent{0pt} % Removes all indentation from paragraphs - comment this line for an assignment with lots of text

%----------------------------------------------------------------------------------------
%	TITLE SECTION
%----------------------------------------------------------------------------------------

\newcommand{\horrule}[1]{\rule{\linewidth}{#1}} % Create horizontal rule command with 1 argument of height
\newcommand{\adenine}{{\tt adenine}~}

\title{	
\normalfont \normalsize
\huge{\tt ADENINE}: A Data ExploratioN pipelINE \\
\horrule{2pt} \\[0.5cm] % Thick bottom horizontal rule
development plan \\ % The assignment title
}

\author{Samuele Fiorini} % Your name

\date{\normalsize\today} % Today's date or a custom date

\begin{document}

\maketitle % Print the title

%----------------------------------------------------------------------------------------
%	PROBLEM 1
%----------------------------------------------------------------------------------------

\section{Introduction and Motivation}

A question that arises at the beginning of almost every new data analysis is
the following:  {\sl are my data relevant with the problem I'm trying to
solve}? \\

The final goal of \adenine is to help its user to have a glimpse of the answer of
this tedious problem. \\

In order to reach this goal \adenine will make wide use of machine learning ad
data mining techniques. The final pipeline will be essentially consist of three steps:

\begin{enumerate}
	
	\item {\bf Preprocessing}: have you ever wondered what would have been changed in 
	your problem if only you had centered your data? Or if you would have scaled down all the
	measures in a certain interval? \adenine will present several classical preprocessing
	procedures, such as: data centering, Min-Max scaling, standardization, normalization, and so on.
	
	\item {\bf Dimensionality reduction}
	
	\item {\bf Clustering}

\end{enumerate}

The final output of \adenine will be an as compact as possible visual and textual representation of 
the results obtained from the pipelines made with each possible combination of the algorithms
available for each step. For instance, referring to a pipeline built as: \\

data normalization  + PCA + K-Means \\

you can find something like:

\begin{itemize}

	\item an output file containing the norm of the original variables (which has
	been divided to in order to coerce all the features in $[0,1]$),

	\item a 2-D or 3-D scatter plot of the data projected along the 2 or 3 principal components
	estimated by means of PCA and the percentage of explained variance associated
	with each component,

	\item a pictorial representation of the clustering results of the data
	obtained with the optimum number of cluster (learned from the data).

\end{itemize}

\subsection{Material for PhD progress}

The study behind the implementation of \adenine  will be useful in terms of
four PhD courses of my first-year work plan:

\begin{enumerate}

	\item {\sl A Machine Learning Crash Course} [DIBRIS] (Odone, Rosasco): \adenine will cover
	a fair number of (mainly unsupervised) machine learning techniques. Hence, this course
	has been fundamental to acquire the statistical learning background needed to become aware of
	the underlying mechanisms of the algorithms.

	\item {\sl Programming Concepts in Python} [DIBRIS] (Tacchella):  I plan to implement \adenine in
	{\tt Python}. Hence,  most of the implementation choices will be made on the basis of the material
	covered in the course.

  	\item {\sl Programming Complex Heterogeneous Parallel Systems} [IMATI]
	(Clematis,   D'Agostino, Danovaro, Galizia) and {the \sl 24th Summer School on
	Parallel Computing} [CINECA] (Erbacci): \adenine will present several {\sl embarrassingly
	parallel workload} as well as several {\sl isolate GPU accelerable} computations. 
	The former PhD course and the latter school will allow me to develop the parallel computing
	attitude I need to implement \adenine in an as optimized as possible way.

\end{enumerate}


\section{Covered Topics}

\subsection{Preprocessing}
\subsection{Dimensionality reduction}
\subsection{Clustering}


%----------------------------------------------------------------------------------------

\end{document}
